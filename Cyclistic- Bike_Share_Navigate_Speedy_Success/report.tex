% Options for packages loaded elsewhere
\PassOptionsToPackage{unicode}{hyperref}
\PassOptionsToPackage{hyphens}{url}
%
\documentclass[
]{article}
\usepackage{amsmath,amssymb}
\usepackage{iftex}
\ifPDFTeX
  \usepackage[T1]{fontenc}
  \usepackage[utf8]{inputenc}
  \usepackage{textcomp} % provide euro and other symbols
\else % if luatex or xetex
  \usepackage{unicode-math} % this also loads fontspec
  \defaultfontfeatures{Scale=MatchLowercase}
  \defaultfontfeatures[\rmfamily]{Ligatures=TeX,Scale=1}
\fi
\usepackage{lmodern}
\ifPDFTeX\else
  % xetex/luatex font selection
\fi
% Use upquote if available, for straight quotes in verbatim environments
\IfFileExists{upquote.sty}{\usepackage{upquote}}{}
\IfFileExists{microtype.sty}{% use microtype if available
  \usepackage[]{microtype}
  \UseMicrotypeSet[protrusion]{basicmath} % disable protrusion for tt fonts
}{}
\makeatletter
\@ifundefined{KOMAClassName}{% if non-KOMA class
  \IfFileExists{parskip.sty}{%
    \usepackage{parskip}
  }{% else
    \setlength{\parindent}{0pt}
    \setlength{\parskip}{6pt plus 2pt minus 1pt}}
}{% if KOMA class
  \KOMAoptions{parskip=half}}
\makeatother
\usepackage{xcolor}
\usepackage[margin=1in]{geometry}
\usepackage{color}
\usepackage{fancyvrb}
\newcommand{\VerbBar}{|}
\newcommand{\VERB}{\Verb[commandchars=\\\{\}]}
\DefineVerbatimEnvironment{Highlighting}{Verbatim}{commandchars=\\\{\}}
% Add ',fontsize=\small' for more characters per line
\usepackage{framed}
\definecolor{shadecolor}{RGB}{248,248,248}
\newenvironment{Shaded}{\begin{snugshade}}{\end{snugshade}}
\newcommand{\AlertTok}[1]{\textcolor[rgb]{0.94,0.16,0.16}{#1}}
\newcommand{\AnnotationTok}[1]{\textcolor[rgb]{0.56,0.35,0.01}{\textbf{\textit{#1}}}}
\newcommand{\AttributeTok}[1]{\textcolor[rgb]{0.13,0.29,0.53}{#1}}
\newcommand{\BaseNTok}[1]{\textcolor[rgb]{0.00,0.00,0.81}{#1}}
\newcommand{\BuiltInTok}[1]{#1}
\newcommand{\CharTok}[1]{\textcolor[rgb]{0.31,0.60,0.02}{#1}}
\newcommand{\CommentTok}[1]{\textcolor[rgb]{0.56,0.35,0.01}{\textit{#1}}}
\newcommand{\CommentVarTok}[1]{\textcolor[rgb]{0.56,0.35,0.01}{\textbf{\textit{#1}}}}
\newcommand{\ConstantTok}[1]{\textcolor[rgb]{0.56,0.35,0.01}{#1}}
\newcommand{\ControlFlowTok}[1]{\textcolor[rgb]{0.13,0.29,0.53}{\textbf{#1}}}
\newcommand{\DataTypeTok}[1]{\textcolor[rgb]{0.13,0.29,0.53}{#1}}
\newcommand{\DecValTok}[1]{\textcolor[rgb]{0.00,0.00,0.81}{#1}}
\newcommand{\DocumentationTok}[1]{\textcolor[rgb]{0.56,0.35,0.01}{\textbf{\textit{#1}}}}
\newcommand{\ErrorTok}[1]{\textcolor[rgb]{0.64,0.00,0.00}{\textbf{#1}}}
\newcommand{\ExtensionTok}[1]{#1}
\newcommand{\FloatTok}[1]{\textcolor[rgb]{0.00,0.00,0.81}{#1}}
\newcommand{\FunctionTok}[1]{\textcolor[rgb]{0.13,0.29,0.53}{\textbf{#1}}}
\newcommand{\ImportTok}[1]{#1}
\newcommand{\InformationTok}[1]{\textcolor[rgb]{0.56,0.35,0.01}{\textbf{\textit{#1}}}}
\newcommand{\KeywordTok}[1]{\textcolor[rgb]{0.13,0.29,0.53}{\textbf{#1}}}
\newcommand{\NormalTok}[1]{#1}
\newcommand{\OperatorTok}[1]{\textcolor[rgb]{0.81,0.36,0.00}{\textbf{#1}}}
\newcommand{\OtherTok}[1]{\textcolor[rgb]{0.56,0.35,0.01}{#1}}
\newcommand{\PreprocessorTok}[1]{\textcolor[rgb]{0.56,0.35,0.01}{\textit{#1}}}
\newcommand{\RegionMarkerTok}[1]{#1}
\newcommand{\SpecialCharTok}[1]{\textcolor[rgb]{0.81,0.36,0.00}{\textbf{#1}}}
\newcommand{\SpecialStringTok}[1]{\textcolor[rgb]{0.31,0.60,0.02}{#1}}
\newcommand{\StringTok}[1]{\textcolor[rgb]{0.31,0.60,0.02}{#1}}
\newcommand{\VariableTok}[1]{\textcolor[rgb]{0.00,0.00,0.00}{#1}}
\newcommand{\VerbatimStringTok}[1]{\textcolor[rgb]{0.31,0.60,0.02}{#1}}
\newcommand{\WarningTok}[1]{\textcolor[rgb]{0.56,0.35,0.01}{\textbf{\textit{#1}}}}
\usepackage{graphicx}
\makeatletter
\def\maxwidth{\ifdim\Gin@nat@width>\linewidth\linewidth\else\Gin@nat@width\fi}
\def\maxheight{\ifdim\Gin@nat@height>\textheight\textheight\else\Gin@nat@height\fi}
\makeatother
% Scale images if necessary, so that they will not overflow the page
% margins by default, and it is still possible to overwrite the defaults
% using explicit options in \includegraphics[width, height, ...]{}
\setkeys{Gin}{width=\maxwidth,height=\maxheight,keepaspectratio}
% Set default figure placement to htbp
\makeatletter
\def\fps@figure{htbp}
\makeatother
\setlength{\emergencystretch}{3em} % prevent overfull lines
\providecommand{\tightlist}{%
  \setlength{\itemsep}{0pt}\setlength{\parskip}{0pt}}
\setcounter{secnumdepth}{-\maxdimen} % remove section numbering
\ifLuaTeX
  \usepackage{selnolig}  % disable illegal ligatures
\fi
\IfFileExists{bookmark.sty}{\usepackage{bookmark}}{\usepackage{hyperref}}
\IfFileExists{xurl.sty}{\usepackage{xurl}}{} % add URL line breaks if available
\urlstyle{same}
\hypersetup{
  pdftitle={How does a bike-share navigate speedy success?},
  pdfauthor={Fábio Ribeiro},
  hidelinks,
  pdfcreator={LaTeX via pandoc}}

\title{How does a bike-share navigate speedy success?}
\author{Fábio Ribeiro}
\date{2023-08-03}

\begin{document}
\maketitle

\hypertarget{introduction}{%
\section{Introduction}\label{introduction}}

Welcome to the Cyclistic bike-share analysis case study. In this
analysis, we will explore the historical bike trip data provided by
Cyclistic and aim to answer key business questions that will guide our
marketing strategy.

\hypertarget{business-problem}{%
\subsection{Business Problem}\label{business-problem}}

The marketing team aims to maximize annual memberships. To achieve this,
we need to analyze the differences between casual riders and annual
members, identify potential conversion opportunities, and design
effective marketing strategies.

\hypertarget{data-acquisition}{%
\subsection{Data Acquisition}\label{data-acquisition}}

We have gathered 10 months of bike trip data from April 2020 to January
2021, available to download
\href{https://divvy-tripdata.s3.amazonaws.com/index.html}{here} combined
them into a single dataset. We have cleaned and preprocessed the data to
ensure its quality.

\begin{Shaded}
\begin{Highlighting}[]
\CommentTok{\# Data Acquisition and Pre{-}processing}

\CommentTok{\# Load required packages}
\FunctionTok{library}\NormalTok{(tidyverse)}
\end{Highlighting}
\end{Shaded}

\begin{verbatim}
## -- Attaching core tidyverse packages ------------------------ tidyverse 2.0.0 --
## v dplyr     1.1.2     v readr     2.1.4
## v forcats   1.0.0     v stringr   1.5.0
## v ggplot2   3.4.2     v tibble    3.2.1
## v lubridate 1.9.2     v tidyr     1.3.0
## v purrr     1.0.1     
## -- Conflicts ------------------------------------------ tidyverse_conflicts() --
## x dplyr::filter() masks stats::filter()
## x dplyr::lag()    masks stats::lag()
## i Use the conflicted package (<http://conflicted.r-lib.org/>) to force all conflicts to become errors
\end{verbatim}

\begin{Shaded}
\begin{Highlighting}[]
\FunctionTok{library}\NormalTok{(lubridate)}
\FunctionTok{library}\NormalTok{(janitor)}
\end{Highlighting}
\end{Shaded}

\begin{verbatim}
## 
## Attaching package: 'janitor'
## 
## The following objects are masked from 'package:stats':
## 
##     chisq.test, fisher.test
\end{verbatim}

\begin{Shaded}
\begin{Highlighting}[]
\CommentTok{\# Load and combine data}
\NormalTok{df1 }\OtherTok{\textless{}{-}} \FunctionTok{read.csv}\NormalTok{(}\StringTok{"./Bike{-}Share{-}Navigate{-}Speedy{-}Success/Data/202004{-}divvy{-}tripdata.csv"}\NormalTok{)}
\NormalTok{df2 }\OtherTok{\textless{}{-}} \FunctionTok{read.csv}\NormalTok{(}\StringTok{"./Bike{-}Share{-}Navigate{-}Speedy{-}Success/Data/202005{-}divvy{-}tripdata.csv"}\NormalTok{)}
\NormalTok{df3 }\OtherTok{\textless{}{-}} \FunctionTok{read.csv}\NormalTok{(}\StringTok{"./Bike{-}Share{-}Navigate{-}Speedy{-}Success/Data/202006{-}divvy{-}tripdata.csv"}\NormalTok{)}
\NormalTok{df4 }\OtherTok{\textless{}{-}} \FunctionTok{read.csv}\NormalTok{(}\StringTok{"./Bike{-}Share{-}Navigate{-}Speedy{-}Success/Data/202007{-}divvy{-}tripdata.csv"}\NormalTok{)}
\NormalTok{df5 }\OtherTok{\textless{}{-}} \FunctionTok{read.csv}\NormalTok{(}\StringTok{"./Bike{-}Share{-}Navigate{-}Speedy{-}Success/Data/202008{-}divvy{-}tripdata.csv"}\NormalTok{)}
\NormalTok{df6 }\OtherTok{\textless{}{-}} \FunctionTok{read.csv}\NormalTok{(}\StringTok{"./Bike{-}Share{-}Navigate{-}Speedy{-}Success/Data/202009{-}divvy{-}tripdata.csv"}\NormalTok{)}
\NormalTok{df7 }\OtherTok{\textless{}{-}} \FunctionTok{read.csv}\NormalTok{(}\StringTok{"./Bike{-}Share{-}Navigate{-}Speedy{-}Success/Data/202010{-}divvy{-}tripdata.csv"}\NormalTok{)}
\NormalTok{df8 }\OtherTok{\textless{}{-}} \FunctionTok{read.csv}\NormalTok{(}\StringTok{"./Bike{-}Share{-}Navigate{-}Speedy{-}Success/Data/202011{-}divvy{-}tripdata.csv"}\NormalTok{)}
\NormalTok{df9 }\OtherTok{\textless{}{-}} \FunctionTok{read.csv}\NormalTok{(}\StringTok{"./Bike{-}Share{-}Navigate{-}Speedy{-}Success/Data/202012{-}divvy{-}tripdata.csv"}\NormalTok{)}
\NormalTok{df10 }\OtherTok{\textless{}{-}} \FunctionTok{read.csv}\NormalTok{(}\StringTok{"./Bike{-}Share{-}Navigate{-}Speedy{-}Success/Data/202101{-}divvy{-}tripdata.csv"}\NormalTok{)}
\NormalTok{bike\_rides }\OtherTok{\textless{}{-}} \FunctionTok{rbind}\NormalTok{(df1, df2, df3, df4, df5, df6, df7, df8, df9, df10)}

\CommentTok{\# Clean data}
\NormalTok{bike\_rides }\OtherTok{\textless{}{-}}\NormalTok{ janitor}\SpecialCharTok{::}\FunctionTok{remove\_empty}\NormalTok{(bike\_rides, }\AttributeTok{which =} \FunctionTok{c}\NormalTok{(}\StringTok{"cols"}\NormalTok{, }\StringTok{"rows"}\NormalTok{))}
\NormalTok{bike\_rides }\OtherTok{\textless{}{-}}\NormalTok{ bike\_rides }\SpecialCharTok{\%\textgreater{}\%} \FunctionTok{distinct}\NormalTok{()}

\CommentTok{\# Clean environment}
\FunctionTok{rm}\NormalTok{(df1, df2, df3, df4, df5, df6, df7, df8, df9, df10)}
\end{Highlighting}
\end{Shaded}

\hypertarget{data-analysis}{%
\subsection{Data Analysis}\label{data-analysis}}

Now that we have prepared the data, we can start analyzing it to gain
insights into user behavior and usage patterns.

\begin{Shaded}
\begin{Highlighting}[]
\DocumentationTok{\#\#\#make a copy for further analysis {-} bike\_rides\_2 is cleaned.}
\NormalTok{bike\_rides\_2 }\OtherTok{\textless{}{-}}\NormalTok{bike\_rides}

\DocumentationTok{\#\#Create ride length, hour and date field}
\NormalTok{bike\_rides\_2}\SpecialCharTok{$}\NormalTok{start\_hour }\OtherTok{\textless{}{-}}\NormalTok{ lubridate}\SpecialCharTok{::}\FunctionTok{hour}\NormalTok{(bike\_rides\_2}\SpecialCharTok{$}\NormalTok{started\_at)}
\NormalTok{bike\_rides\_2}\SpecialCharTok{$}\NormalTok{end\_hour }\OtherTok{\textless{}{-}}\NormalTok{ lubridate}\SpecialCharTok{::}\FunctionTok{hour}\NormalTok{(bike\_rides\_2}\SpecialCharTok{$}\NormalTok{ended\_at)}
\NormalTok{bike\_rides\_2}\SpecialCharTok{$}\NormalTok{start\_date }\OtherTok{\textless{}{-}}\NormalTok{ lubridate}\SpecialCharTok{::}\FunctionTok{date}\NormalTok{(bike\_rides\_2}\SpecialCharTok{$}\NormalTok{started\_at)}
\NormalTok{bike\_rides\_2}\SpecialCharTok{$}\NormalTok{end\_date }\OtherTok{\textless{}{-}}\NormalTok{ lubridate}\SpecialCharTok{::}\FunctionTok{date}\NormalTok{(bike\_rides\_2}\SpecialCharTok{$}\NormalTok{ended\_at)}
\NormalTok{bike\_rides\_2}\SpecialCharTok{$}\NormalTok{ride\_length }\OtherTok{\textless{}{-}}\FunctionTok{difftime}\NormalTok{(bike\_rides\_2}\SpecialCharTok{$}\NormalTok{ended\_at,bike\_rides\_2}\SpecialCharTok{$}\NormalTok{started\_at, }\AttributeTok{units=}\StringTok{"mins"}\NormalTok{)}

\DocumentationTok{\#\#\#The raw data also contains unneeded data such as ride IDs, station IDs, and}
\DocumentationTok{\#\#latitude and longitude coordinates ,}
\CommentTok{\#remove rows where ride\_length is \textless{}= to zero}
\NormalTok{bike\_rides\_2 }\OtherTok{\textless{}{-}}\NormalTok{ bike\_rides\_2[}\SpecialCharTok{!}\NormalTok{(bike\_rides\_2}\SpecialCharTok{$}\NormalTok{ride\_length }\SpecialCharTok{\textless{}=}\DecValTok{0}\NormalTok{),]}
\CommentTok{\#remove unneeded columns}
\NormalTok{bike\_rides\_2 }\OtherTok{\textless{}{-}}\NormalTok{ bike\_rides\_2 }\SpecialCharTok{\%\textgreater{}\%}  
  \FunctionTok{select}\NormalTok{(}\SpecialCharTok{{-}}\FunctionTok{c}\NormalTok{(ride\_id, start\_station\_id, end\_station\_id, start\_lat, start\_lng, end\_lat, end\_lng))}

\DocumentationTok{\#\#Rename column name "member\_casual" since there is two possible values, casual and member.}
\NormalTok{bike\_rides\_2 }\OtherTok{\textless{}{-}}\NormalTok{ bike\_rides\_2 }\SpecialCharTok{\%\textgreater{}\%} 
  \FunctionTok{rename}\NormalTok{(}\AttributeTok{member\_type =}\NormalTok{ member\_casual)}

\CommentTok{\#convert ride\_length to numeric}
\NormalTok{bike\_rides\_2}\SpecialCharTok{$}\NormalTok{ride\_length }\OtherTok{\textless{}{-}} \FunctionTok{as.numeric}\NormalTok{(}\FunctionTok{as.character}\NormalTok{(bike\_rides\_2}\SpecialCharTok{$}\NormalTok{ride\_length))}
\end{Highlighting}
\end{Shaded}

We will examine the hourly usage patterns of casual riders and annual
members.

\begin{Shaded}
\begin{Highlighting}[]
\DocumentationTok{\#\#\#PLOT {-} number of rides per hour}
\NormalTok{bike\_rides\_2 }\SpecialCharTok{\%\textgreater{}\%} \FunctionTok{count}\NormalTok{(start\_hour, }\AttributeTok{sort=} \ConstantTok{TRUE}\NormalTok{) }\SpecialCharTok{\%\textgreater{}\%}
  \FunctionTok{ggplot}\NormalTok{()}\SpecialCharTok{+}
  \FunctionTok{geom\_line}\NormalTok{((}\FunctionTok{aes}\NormalTok{(}\AttributeTok{x=}\NormalTok{start\_hour,}\AttributeTok{y=}\NormalTok{n)))}\SpecialCharTok{+}
  \FunctionTok{labs}\NormalTok{(}\AttributeTok{title=}\StringTok{"Count of Bike Rides by Hour: Previous 10 months"}\NormalTok{, }\AttributeTok{x=}\StringTok{"Start Hours"}\NormalTok{, }\AttributeTok{y=}\StringTok{"Number of Rides"}\NormalTok{)}
\end{Highlighting}
\end{Shaded}

\includegraphics{report_files/figure-latex/unnamed-chunk-4-1.pdf}

\begin{Shaded}
\begin{Highlighting}[]
\CommentTok{\#converts values from scientific notation }
\FunctionTok{options}\NormalTok{(}\AttributeTok{scipen =} \DecValTok{999}\NormalTok{)}
\end{Highlighting}
\end{Shaded}

As we can see, there is a obvious decline during night hours, with
maximum volume at around 17h.

The total number of rides in this dataset was 3200708.

\begin{Shaded}
\begin{Highlighting}[]
\FunctionTok{nrow}\NormalTok{(bike\_rides\_2)}
\end{Highlighting}
\end{Shaded}

\begin{verbatim}
## [1] 3200708
\end{verbatim}

There was 1332801 casual rides and 1867907 member during this period.

\begin{Shaded}
\begin{Highlighting}[]
\DocumentationTok{\#\#count by member type}
\NormalTok{bike\_rides\_2 }\SpecialCharTok{\%\textgreater{}\%} \FunctionTok{count}\NormalTok{(member\_type)}
\end{Highlighting}
\end{Shaded}

\begin{verbatim}
##   member_type       n
## 1      casual 1332801
## 2      member 1867907
\end{verbatim}

Regarding the type of bike, docked bikes takes the front with 2530905
rides, followed by electric bikes with 537492 rides, while classic bikes
coming last at 132311 bike rides.

\begin{Shaded}
\begin{Highlighting}[]
\CommentTok{\#total rides by bike type}
\NormalTok{bike\_rides\_2 }\SpecialCharTok{\%\textgreater{}\%}
  \FunctionTok{group\_by}\NormalTok{(rideable\_type) }\SpecialCharTok{\%\textgreater{}\%} 
  \FunctionTok{count}\NormalTok{(rideable\_type)}
\end{Highlighting}
\end{Shaded}

\begin{verbatim}
## # A tibble: 3 x 2
## # Groups:   rideable_type [3]
##   rideable_type       n
##   <chr>           <int>
## 1 classic_bike   132311
## 2 docked_bike   2530905
## 3 electric_bike  537492
\end{verbatim}

Here is a plot to better visualize this data.

\begin{Shaded}
\begin{Highlighting}[]
\NormalTok{bike\_rides\_2 }\SpecialCharTok{\%\textgreater{}\%}
  \FunctionTok{group\_by}\NormalTok{(rideable\_type, member\_type) }\SpecialCharTok{\%\textgreater{}\%}
\NormalTok{  dplyr}\SpecialCharTok{::}\FunctionTok{summarize}\NormalTok{(}\AttributeTok{count\_trips =} \FunctionTok{n}\NormalTok{()) }\SpecialCharTok{\%\textgreater{}\%}  
  \FunctionTok{ggplot}\NormalTok{(}\FunctionTok{aes}\NormalTok{(}\AttributeTok{x=}\NormalTok{rideable\_type, }\AttributeTok{y=}\NormalTok{count\_trips, }\AttributeTok{fill=}\NormalTok{member\_type, }\AttributeTok{color=}\NormalTok{member\_type)) }\SpecialCharTok{+}
  \FunctionTok{geom\_bar}\NormalTok{(}\AttributeTok{stat=}\StringTok{\textquotesingle{}identity\textquotesingle{}}\NormalTok{, }\AttributeTok{position=}\StringTok{\textquotesingle{}dodge\textquotesingle{}}\NormalTok{) }\SpecialCharTok{+}
  \FunctionTok{theme\_bw}\NormalTok{()}\SpecialCharTok{+}
  \FunctionTok{labs}\NormalTok{(}\AttributeTok{title=}\StringTok{"Number of Trips by Bicycle Type"}\NormalTok{, }\AttributeTok{x=}\StringTok{"Bicycle Type"}\NormalTok{, }\AttributeTok{y=}\StringTok{"Number of Rides"}\NormalTok{)}
\end{Highlighting}
\end{Shaded}

\begin{verbatim}
## `summarise()` has grouped output by 'rideable_type'. You can override using the
## `.groups` argument.
\end{verbatim}

\includegraphics{report_files/figure-latex/unnamed-chunk-8-1.pdf}

The average ride during this period was 28.37 minutes while the median
was 14.82 minutes.Note that the minimum and maximum values, 0,02 minutes
and 58720.03 are not realistic so they are probably due to some external
factor like data upload or some faulty equipment.

\begin{Shaded}
\begin{Highlighting}[]
\CommentTok{\#mean length of ride}
\FunctionTok{summary}\NormalTok{(bike\_rides\_2}\SpecialCharTok{$}\NormalTok{ride\_length)}
\end{Highlighting}
\end{Shaded}

\begin{verbatim}
##     Min.  1st Qu.   Median     Mean  3rd Qu.     Max. 
##     0.02     8.07    14.82    28.37    27.03 58720.03
\end{verbatim}

While we got already some hints for when we can enroll in a marketing
campaign, like send a promotion to subscribe to casual users at the
evening, that was showed when the most users will ride our bikes, and
also focusing on docked bikes, let's see what data tells us regarding
weekdays.

\begin{Shaded}
\begin{Highlighting}[]
\CommentTok{\#create column day\_of\_week }
\NormalTok{bike\_rides\_2}\SpecialCharTok{$}\NormalTok{day\_of\_week }\OtherTok{\textless{}{-}} \FunctionTok{wday}\NormalTok{(bike\_rides\_2}\SpecialCharTok{$}\NormalTok{started\_at)}



\CommentTok{\#plot number of rides by day of week (1 is Sunday)}
\NormalTok{bike\_rides\_2 }\SpecialCharTok{\%\textgreater{}\%} 
  \FunctionTok{group\_by}\NormalTok{(member\_type, day\_of\_week) }\SpecialCharTok{\%\textgreater{}\%}
\NormalTok{  dplyr}\SpecialCharTok{::}\FunctionTok{summarize}\NormalTok{(}\AttributeTok{count\_trips =} \FunctionTok{n}\NormalTok{()) }\SpecialCharTok{\%\textgreater{}\%}  
  \FunctionTok{ggplot}\NormalTok{(}\FunctionTok{aes}\NormalTok{(}\AttributeTok{x=}\NormalTok{ day\_of\_week, }\AttributeTok{y=}\NormalTok{count\_trips, }\AttributeTok{fill=}\NormalTok{member\_type, }\AttributeTok{color=}\NormalTok{member\_type)) }\SpecialCharTok{+}
  \FunctionTok{geom\_bar}\NormalTok{(}\AttributeTok{stat=}\StringTok{\textquotesingle{}identity\textquotesingle{}}\NormalTok{, }\AttributeTok{position =} \StringTok{\textquotesingle{}dodge\textquotesingle{}}\NormalTok{) }\SpecialCharTok{+}
  \FunctionTok{scale\_x\_continuous}\NormalTok{(}\AttributeTok{breaks =} \DecValTok{1}\SpecialCharTok{:}\DecValTok{7}\NormalTok{, }\AttributeTok{labels =} \FunctionTok{c}\NormalTok{(}\StringTok{"1"}\NormalTok{, }\StringTok{"2"}\NormalTok{, }\StringTok{"3"}\NormalTok{, }\StringTok{"4"}\NormalTok{, }\StringTok{"5"}\NormalTok{, }\StringTok{"6"}\NormalTok{, }\StringTok{"7"}\NormalTok{)) }\SpecialCharTok{+} 
  \FunctionTok{theme\_bw}\NormalTok{()}\SpecialCharTok{+}
  \FunctionTok{labs}\NormalTok{(}\AttributeTok{title =}\StringTok{"Number of Rides by Day of Week"}\NormalTok{, }\AttributeTok{x =} \StringTok{"Day of Week"}\NormalTok{, }\AttributeTok{y =} \StringTok{"Number of Rides"}\NormalTok{)}
\end{Highlighting}
\end{Shaded}

\begin{verbatim}
## `summarise()` has grouped output by 'member_type'. You can override using the
## `.groups` argument.
\end{verbatim}

\includegraphics{report_files/figure-latex/unnamed-chunk-10-1.pdf}

Our most popular day is Saturday and least popular is monday, while most
casual members use our bikes on weekends.

What about popular stations for casual riders?

\begin{Shaded}
\begin{Highlighting}[]
\CommentTok{\#Find popular start station for casual}
\NormalTok{bike\_rides\_2 }\SpecialCharTok{\%\textgreater{}\%}
  \FunctionTok{group\_by}\NormalTok{(member\_type,start\_station\_name) }\SpecialCharTok{\%\textgreater{}\%}
\NormalTok{  dplyr}\SpecialCharTok{::}\FunctionTok{summarise}\NormalTok{(}\AttributeTok{number\_of\_ride =} \FunctionTok{n}\NormalTok{()) }\SpecialCharTok{\%\textgreater{}\%}
  \FunctionTok{filter}\NormalTok{(start\_station\_name }\SpecialCharTok{!=} \StringTok{""}\NormalTok{, }\StringTok{"casual"} \SpecialCharTok{==}\NormalTok{ member\_type) }\SpecialCharTok{\%\textgreater{}\%}
  \FunctionTok{arrange}\NormalTok{(}\SpecialCharTok{{-}}\NormalTok{number\_of\_ride) }\SpecialCharTok{\%\textgreater{}\%}
  \FunctionTok{head}\NormalTok{(}\AttributeTok{n=}\DecValTok{5}\NormalTok{) }\SpecialCharTok{\%\textgreater{}\%}
  \FunctionTok{select}\NormalTok{(}\SpecialCharTok{{-}}\NormalTok{member\_type)}
\end{Highlighting}
\end{Shaded}

\begin{verbatim}
## `summarise()` has grouped output by 'member_type'. You can override using the
## `.groups` argument.
## Adding missing grouping variables: `member_type`
\end{verbatim}

\begin{verbatim}
## # A tibble: 5 x 3
## # Groups:   member_type [1]
##   member_type start_station_name         number_of_ride
##   <chr>       <chr>                               <int>
## 1 casual      Streeter Dr & Grand Ave             24599
## 2 casual      Lake Shore Dr & Monroe St           18689
## 3 casual      Millennium Park                     17960
## 4 casual      Theater on the Lake                 14143
## 5 casual      Indiana Ave & Roosevelt Rd          12801
\end{verbatim}

``Streeter Dr \& Grave'', ``Lake Shore Dr'' and ``Millenium Park'' is
our most popular stations for casual riders.

\hypertarget{recommendations}{%
\subsection{Recommendations}\label{recommendations}}

\begin{enumerate}
\def\labelenumi{\arabic{enumi}.}
\item
  \textbf{Targeted Promotions:} Leverage the hourly usage patterns to
  target specific timeframes for marketing promotions. Focus on the peak
  usage hours, particularly around 17:00, when the number of rides is at
  its highest. Consider sending promotions or discounts to casual riders
  during these hours to encourage more rides and potentially convert
  them to annual members.
\item
  \textbf{Weekend Campaigns:} Since Saturdays are the most popular days
  for bike rides and weekends generally see higher casual rider
  activity, design marketing campaigns that specifically target
  weekends. Offering special weekend-only membership deals or incentives
  can attract more riders during these days.
\item
  \textbf{Bike Type Emphasis:} Concentrate marketing efforts on docked
  bikes, as they are the most frequently used bike type. Highlight the
  benefits of docked bikes in marketing materials, showcasing their
  availability and convenience, which might encourage more riders to
  choose this type.
\item
  \textbf{Weekday Strategies:} Implement strategies to boost weekday
  ridership, especially on Mondays when the usage is lower. Offer
  special deals or incentives for rides on Mondays to attract more
  riders on this typically less busy day.
\item
  \textbf{Station Improvement:} Focus on popular start stations like
  ``Streeter Dr \& Grave,'' ``Lake Shore Dr,'' and ``Millennium Park.''
  Ensure these stations are well-maintained, easily accessible, and have
  a sufficient number of available bikes. Enhancing the user experience
  at these stations can positively impact customer satisfaction and
  loyalty.
\end{enumerate}

\hypertarget{conclusion}{%
\subsection{Conclusion}\label{conclusion}}

In this analysis, we explored Cyclistic's bike trip data and gained
insights into user behavior. We identified key differences between
casual riders and annual members and proposed strategies to increase
annual memberships.

\end{document}
